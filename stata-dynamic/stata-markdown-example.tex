\PassOptionsToPackage{unicode=true}{hyperref} % options for packages loaded elsewhere
\PassOptionsToPackage{hyphens}{url}
%
\documentclass[]{article}
\usepackage{stata}
\usepackage{lmodern}
\usepackage{amssymb,amsmath}
\usepackage{ifxetex,ifluatex}
\usepackage{fixltx2e} % provides \textsubscript
\ifnum 0\ifxetex 1\fi\ifluatex 1\fi=0 % if pdftex
  \usepackage[T1]{fontenc}
  \usepackage[utf8]{inputenc}
  \usepackage{textcomp} % provides euro and other symbols
\else % if luatex or xelatex
  \usepackage{unicode-math}
  \defaultfontfeatures{Ligatures=TeX,Scale=MatchLowercase}
\fi
% use upquote if available, for straight quotes in verbatim environments
\IfFileExists{upquote.sty}{\usepackage{upquote}}{}
% use microtype if available
\IfFileExists{microtype.sty}{%
\usepackage[]{microtype}
\UseMicrotypeSet[protrusion]{basicmath} % disable protrusion for tt fonts
}{}
\IfFileExists{parskip.sty}{%
\usepackage{parskip}
}{% else
\setlength{\parindent}{0pt}
\setlength{\parskip}{6pt plus 2pt minus 1pt}
}
\usepackage{hyperref}
\hypersetup{
            pdftitle={Stata Markdown Tutorial},
            pdfauthor={Cyrus Samii},
            pdfborder={0 0 0},
            breaklinks=true}
\urlstyle{same}  % don't use monospace font for urls
\usepackage{graphicx,grffile}
\makeatletter
\def\maxwidth{\ifdim\Gin@nat@width>\linewidth\linewidth\else\Gin@nat@width\fi}
\def\maxheight{\ifdim\Gin@nat@height>\textheight\textheight\else\Gin@nat@height\fi}
\makeatother
% Scale images if necessary, so that they will not overflow the page
% margins by default, and it is still possible to overwrite the defaults
% using explicit options in \includegraphics[width, height, ...]{}
\setkeys{Gin}{width=\maxwidth,height=\maxheight,keepaspectratio}
\setlength{\emergencystretch}{3em}  % prevent overfull lines
\providecommand{\tightlist}{%
  \setlength{\itemsep}{0pt}\setlength{\parskip}{0pt}}
\setcounter{secnumdepth}{0}
% Redefines (sub)paragraphs to behave more like sections
\ifx\paragraph\undefined\else
\let\oldparagraph\paragraph
\renewcommand{\paragraph}[1]{\oldparagraph{#1}\mbox{}}
\fi
\ifx\subparagraph\undefined\else
\let\oldsubparagraph\subparagraph
\renewcommand{\subparagraph}[1]{\oldsubparagraph{#1}\mbox{}}
\fi

% set default figure placement to htbp
\makeatletter
\def\fps@figure{htbp}
\makeatother

\usepackage{multicol}
\usepackage{tabularx}
\usepackage{booktabs}
\usepackage{lscape}

\title{Stata Markdown Tutorial}
\author{Cyrus Samii}
\date{January 2019}

\begin{document}
\maketitle

\hypertarget{overview}{%
\section{Overview}\label{overview}}

Here are some notes and examples for using Stata Markdown from German
Rodriguez. For instructions on installation and dependencies, refer to
the Stata Markdown website.

I give examples of some things we might want to do in social science
related projects.

\hypertarget{markdown}{%
\section{Markdown}\label{markdown}}

Markdown is a simple markup language that, through Pandoc, can be
rendered in a variety of formats, including pdf (via tex), html, or
docx. If you are used to writing latex or html, then markdown will be
easy, since it admits a lot of the syntax used in those languages.

There are lots of cheatsheets out there, such as:

\url{https://github.com/adam-p/markdown-here/wiki/Markdown-Cheatsheet}

Lots of things are done very simply in Markdown. E.g., here is a
numbered list:

\begin{enumerate}
\def\labelenumi{\arabic{enumi}.}
\tightlist
\item
  Foo
\item
  Foo 2
\item
  Foo 3
\end{enumerate}

The header of this document is a YAML header for Markdown, which
contains meta instructions for the Markdown-\textgreater{}Pandoc
compilation.

\hypertarget{workflow}{%
\section{Workflow}\label{workflow}}

The way I work is to type into this document and then compile by running
the requisite commands that I have put into a separate .do file called
``stata-markdown-example-do.do''. That way, I can load the various
compilation options (that is, the options to the \texttt{markstat}
function in a way that I can easily recall them later. Using the
\texttt{do} button in the Stata .do file editor gives me one button
compilation. I also have my commands to set the working directory and
also load in dependencies (e.g., the \texttt{stata.sty} file needed to
compile to PDF).

I may also have another Stata .do file that I use as a scratch pad for
working out the kinks of the Stata code that I then insert as code
chunks into this document.

\hypertarget{simple-script-example}{%
\section{``Simple Script'' Example}\label{simple-script-example}}

Here we replicate the simple example from German Rodriguez's ``Simple
Script'' example, tweaking a few things to make some additional points.

First, we read the fuel efficiency data that is shipped with Stata:

\begin{stlog}
. sysuse auto, clear
(1978 Automobile Data)
\end{stlog}

To study how fuel efficiency depends on weight it is useful to transform
the dependent variable from ``miles per gallon'' to ``gallons per 100
miles'':

\begin{stlog}
. gen gphm = 100/mpg
\end{stlog}

We can then plot the relationship. We will run this code in a manner
that is not echoed in the resulting output file (PDF, docx, etc.).

\begin{stlog}


{\smallskip}

\end{stlog}

\begin{figure}
\centering
\includegraphics[width=0.75\linewidth]{auto.png}
\caption{Fuel Efficiency}
\end{figure}

\hypertarget{regression-table}{%
\section{Regression table}\label{regression-table}}

Something that we frequently need to do is to report regression tables.
We can use the \texttt{esttab} function in Stata and insert its output
here:

\begin{stlog}

{\smallskip}

{\smallskip}

\end{stlog}

\begin{center}
{
\def\sym#1{\ifmmode^{#1}\else\(^{#1}\)\fi}
\begin{tabular}{l*{1}{c}}
\hline\hline
                    &\multicolumn{1}{c}{(1)}\\
                    &\multicolumn{1}{c}{gphm}\\
\hline
Weight (lbs.)       &        0.00\sym{***}\\
                    &      (0.00)         \\
[1em]
Constant            &        0.77\sym{*}  \\
                    &      (0.33)         \\
\hline
Observations        &          74         \\
r2                  &        0.73         \\
\hline\hline
\multicolumn{2}{l}{\footnotesize Standard errors in parentheses}\\
\multicolumn{2}{l}{\footnotesize \sym{*} \(p<0.05\), \sym{**} \(p<0.01\), \sym{***} \(p<0.001\)}\\
\end{tabular}
}

\end{center}

(If you look at the Stata Markdown .stmd file, you will see that I used
tex commands to insert the regression table and center it.)

\end{document}
